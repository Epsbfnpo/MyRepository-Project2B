\documentclass[12pt, a4paper]{article}
\usepackage{graphicx}
\usepackage{color}
\hyphenpenalty=5000
\tolerance=1000

\begin{document}
	\begin{titlepage}
		\begin{center}

			\begin{center}
				\begin{large}
					\begin{tabular}{cc}

					\end{tabular}
				\end{large}
			\end{center}
			
			\textsc{\large Info1111 RE07-01}

			
		\end{center}
	\end{titlepage}
		
	\clearpage

	\tableofcontents

	\clearpage
	\setcounter{page}{1}
	

	\section{Introduction}

	\clearpage
	
	\section{Major Allocation}

	\clearpage
	
	\section{Recommendations}
	\subsection{Computer Science}

	\clearpage
	
	\subsection{Data Science}

	\clearpage
	
	\subsection{Information Systems}

	\clearpage
	
	\subsection{Software Development}
	
	\subsubsection {HPSC 1001}

	\begin{figure}[h]
	\centering
	\includegraphics[width=8cm,height=5cm]{Science_and_Philosophy}
	\caption{Sciece and Philosophy}
	\end{figure}

	\paragraph {Brief Introduction}

What distinguishes creationism from evolutionary theory, or astrology from astronomy? Can we have good reason to believe that our current scientific theories represent the world "as it really is"? For those people who are immersed in philosophical thinking and reflection, such questions are bound to emerge often in their mind. As students of Advanced Computing, I do admit that learning how to master programming languages, structures,  algorithm and some other basic elements in computing is undoubtedly the top priority. However, we should not rest content with these technologies. Why does computing matter? What is computing’s true features? What are statistics, software and information, if we evaluate them at the level of human development? These questions are really worth studying and once we find appropriate answers, it will enhance our understanding of computing and assist us to innovate and reform in this field.

	\paragraph {Which field it pertains to?}

This elective is from an academic unit named ‘History and Philosophy of Science Academic Operations' according to USYD’s standard of classification. Generally, it is supposed to be classified in the field of History and Philosophy.

	\paragraph {What core knowledge it has?} 

In this unit, we can acquire some knowledge below (For this unit is a completely liberal one, there is not any corresponding technology, contrary to some scientific units).
 
\centerline {\textcolor{red}{Basic Introduction}}

\centerline {\textcolor{red}{Empiricism}}

This is the most basic part in this unit, mainly introducing the structure and form. Empiricism is the first academic definition we will meet. We will deeply study its content, including development history and methodology, and the relationship to Rationalism.
 
\centerline {\textcolor{red}{Logical Positivism}}

\centerline {\textcolor{red}{Problems of Confirmation}}

\centerline {\textcolor{red}{Karl Popper's Falsificationism}}

This is the chapter of Positivism, containing the definition, problems in methodology and famous views. We will deeply study the stage of the development of human reason (Positivism is the highest one), the concrete process while confirmation and the fantastic view ‘Scientific theories are falsifiable in principle.’ raised by Karl Popper\footnote{(28 July 1902 – 17 September 1994) an Austrian-British philosopher, academic and social commentator, known for his rejection of the classical inductivist views on the scientific method in favour of empirical falsification.}. 

\centerline {\textcolor{red}{The Problem of Demarcation}}

\centerline {\textcolor{red}{The Case of Creation Science}}

\centerline {\textcolor{red}{Thomas Kuhn's Revolution}}

\centerline {\textcolor{red}{Responses to Kuhn}}

This is the chapter of demarcation, containing problems, typical cases and a famous revolutionary theory along with its response. We will deeply study how to demarcate objects, especially science in a professional and peer-reviewed way. Moreover, we will learn ‘Science consists of a series of paradigms separated by abrupt scientific revolutions.’, a theory raised by Thomas Kuhn\footnote{(July 18, 1922 – June 17, 1996) an American philosopher of science, known for his 1962 book The Structure of Scientific Revolutions in both academic and popular circles.} and its corresponding responses. 

\centerline {\textcolor{red}{Sociology of Scientific Knowledge}}

\centerline {\textcolor{red}{Values, Objectivity and Social Critiques of Science}}

The scientific elements begin to ascend in this chapter. This is the chapter of sociology. We will focus on the basic attributes of science, containing values, objectivity and social critiques, the role of values in science and sociological approaches to understanding science. Claim that science cannot really be distinguished from other approaches to knowledge is also the significant study content in this chapter. 

\centerline {\textcolor{red}{Relativism and Anti-realism}}

\centerline {\textcolor{red}{Realism and Experiment}}

\centerline {\textcolor{red}{Scientific Explanation}}

This is the chapter of realism. We will learn how to distinguish correct methodologies in experiments and explanations form wrong ones. This is of vital essence for our future development in scientific fields. 

	\paragraph {What abilities it can improve?}

	\begin{itemize}
	  \item Be capable of understanding philosophical and historical discussions of science and critically assess arguments in these areas.
	  \item Be capable of writing clear and well-organized responses to philosophical and historical discussions of science, and develop your own views on these issues (which is a liberal learning outcome).
	  \item Be capable of relating general philosophical and historical ideas about science to particular examples of scientific work in a gesture to rapidly form a complete guidance for us to finish the work.
	\end{itemize}

	\paragraph {How it helps the major?}

This unit is not particularly designed for students who major in Software Development, but it teaches the guiding ideology when we are conducting research into computing. As the famous scientist Descartes\marginpar{\textcolor{blue}{\tiny French philosopher, mathematician, and scientist}} stated, if knowledge is a tree, philosophy will be the root and science will be the branch. Science is appropriate only when it adhere to the guidance of philosophy. Hence, acquaintance with philosophy will take us to a higher platform; there we are capable of getting a bird's-eye view of science’s true features. Moreover, concentrated on Software Development, it is a practical field, because all the objects are attained from concrete social demand and all the responses are oriented towards the society. Hence, the content about Sociology of Science, Realism, Demarcation and other scientific elements works. 

	\paragraph {How it helps the career?}

As I stated just now, this unit is extremely helpful in our future career. No matter what we are about to do, scientific research, IT enterprise management, or professional programming, we need to have a clear understanding of what we are doing and why we are doing in this way, rather than solely mastering cutting-edge technology without knowing the meaning of our course. 

	\paragraph {What opportunities it can bring for us in the future?}

In the future, rapidly increasing technology is an inevitable trend. While some technology may well be confronted with dispute, studying this unit offering us an opportunity to demarcate brand new technology and distinguish the potential ones from mortal ones under the guidance of philosophic ideas.

	\clearpage
	
	\subsubsection {DECO 1006}

	\begin{figure}[h]
	\centering
	\includegraphics[width=8cm,height=5cm]{Design}
	\caption{Design}
	\end{figure}

	\paragraph {Brief Introduction}

Have you ever paid attention to categories of design? If so, have you ever focused on human-centred spirits? What would occur if these two definitions combine together? For people who are not involved in design, they may regard design as a complete art. However, design also needs the support of precise theories and rigorous analysis. Real design is the design of interactive technologies and environments, which requires user research, ideation, prototyping and user evaluation.It provides us with the principles, processes and tools that are used in commercial design projects. We can learn to build empathy with users, identify the problem space, develop design concepts and persuasively communicate design proposals with an emphasis on the user experience through visual storytelling.

	\paragraph {Which field it pertains to?}

This elective is from an academic unit named ‘Design Lab' according to USYD’s standard of classification. Generally, it is supposed to be classified in the combined field of art and technology.

	\paragraph {What core knowledge it has?} 

In this unit, we can acquire some knowledge and technology below.
 
\centerline {\textcolor{red}{Basic Introduction}}

\centerline {\textcolor{red}{Visual Thinking}}

\centerline {\textcolor{red}{Design as a Process}}

This is the most basic part in this unit, mainly introducing the visual thinking and the definition of "design as a process". Visual thinking is of vital essence in the whole unit, often described as seeing words as a series of pictures. One of the ultimate goals of this unit is to design as a real picture thinkers\footnote{"Real picture thinkers" are those who use visual thinking almost to the exclusion of other kinds of thinking, make up a smaller percentage of the population.According to Kreger Silverman, three tenths of the general population who use visual/spatial thinking, only a small percentage would use this style over and above all other forms of thinking, and can be said to be "real picture thinkers"}.
 
\centerline {\textcolor{red}{User Research}}

\centerline {\textcolor{red}{Data Synthesis}}

This is the chapter of User Research, containing the research and the subsequent data synthesis. We will deeply study the concrete process of the user research, including deciding research objects, designing user factor frameworks, analyzing feasibility and designing questionnaires. Moreover, the following data synthesis is the conclusion and summary of the former research, and the foundation of the future design evaluation. 

\centerline {\textcolor{red}{Generating Ideas}}

\centerline {\textcolor{red}{Prototyping}}

\centerline {\textcolor{red}{Design Evaluation}}

This is the chapter of Design Evaluation, containing generating ideas, prototyping and the evaluation. We will deeply study how to generate ideas, especially when information and data we possess is not complete, create the prototypes of the design and evaluate design in several aspects, including the readability of information, artistic styles and so forth. 

\centerline {\textcolor{red}{TBC}}

\centerline {\textcolor{red}{Future of design}}

This chapter enables us to complete the design and provides us with suggestions about details in TBC.

	\paragraph {What abilities it can improve?}

	\begin{itemize}
	  \item Engage in contextual inquiry to identify the need for a design
	  \item Show competence in design ideation
	  \item Communicate ideas and concepts visually
	  \item Apply knowledge of the psychology of user experience to designing interactive systems
	  \item Describe and explain activities associated with a design project
	  \item Reflect upon and critique design activities using appropriate language
	\end{itemize}

	\paragraph {How it helps the major?}

This unit has strong relationship with students who major in Software Development, for software development is actually a process of design. Codes and algorithm matter, but they are all dependent on the initial creativity. Only when developers acquaint the demand and general idea can they carry on typing codes and fulfil the framework in a computing aspect. Hence, acquaintance with design will allow us to self-develop software, no more support from the designers, which means from the design idea of software to the computing structure of that, all this process can be dealt with one person or computational team.

	\paragraph {How it helps the career?}

As I stated just now, this unit is extremely helpful in our future career. As long as we develop software, design matters, because if we only know how to codes, it would be extremely likely that we be weeded out by artificial intelligence. The design idea is what we human are expert at and what AI would hardly get to know.

	\paragraph {What opportunities it can bring for us in the future?}

In the future, acquaintance with design will allow us to self-develop software, no more support from the designers, which means from the design idea of software to the computing structure of that, all this process can be dealt with one person or computational team.

	\clearpage

	\section{Contributions}

	\clearpage
	

	\bibliography{refs}
	
	
\end{document} 